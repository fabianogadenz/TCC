\chapter{Introdução}\label{ch:intro}

As instruções fazem parte do nosso aprendizado e nos levam a efetuarmos atitude corretas. Para evitar sérias consequuencias ao usuário, a correta realizacão de uma tarefa depende da legibilidade das informações fornecidas e da qualidade da representação gráfica das mesmas \cite{FUJITA2007}.


No caso dos medicamentos, as informações descritas na bula são imprescindíveis para o ser humano, pois desempenham um papel fundamental durante o processo de tratamento do mesmo. Segundo Fujita & Spinillo \cite{FUJITA2006}, a apresentação gráfica do conteúdo informal nas bulas de remédios influencia sua leitura e compreensão. Deficiências tanto ao nível de conteúdo com uma linguagem muito formal, quanto na apresentação gráfica das informações em bulas como por exemplo a pequena fonte encontrada, pode ser levado ao mau uso de medicamentos, comprometendo a saúde e até acarretando sérias consequencias na saúde do indivíduo, principalmente para os idosos, muitas vezes com limitações motoras ou limitados de informação, devido a idade. 



Considerando a importância da leitura das bulas e da difícil compreensão de seu conteúdo para o uso correto e seguro de medicamentos, o presente trabalho visa identificar padrões de imagens com técnicas de processamento de imagem (\sigla{PDI}{Processamento de Imagem})  e um estudo entre métodos de reconhecimento ótico de caracteres (\sigla{OCR}{Reconhecimento Ótico de Caracteres}) embarcados no celular, com intenção de melhorar o poder acertivo de métodos de OCR com técnicas de PDI para corrigir a perspectiva da imagem, com a finalidade de identificar o medicamento através de uma foto capturada por meio de um telefone celular da embalagem do medicamento.




\section{Objetivo Geral e específico}
Voltado ao problema da difícil leitura de bula  de medicamento da população brasileira, o objetivo deste trabalho é a apresentação de uma proposta de arquitetura baseada em métodos de processamento de imagem, que possa contribuir para a leitura de métodos OCR processados diretamente pelo aparelho celular.
Este trabalho apresenta um aplicativo \textit{mobile} envolvendo as tecnologias de PDI, OCR com processamento inteiramente no dispositivo móvel. Entre essa proposta apresentada, é preciso desenvolver tais técnicas:

\begin{itemize}
  \item Criar uma base de imagens de bulas e caixas de medicamentos.
  \item Métodos de pré processamentos, como por exemplo, técnicas de iluminação, redução de ruídos e perspectiva da imagem capturada.
  \item Comparar o OCR da Google com e sem as técnicas de processamento de imagem.
  \item Apresentar informações detalhadas do medicamento inspecionado na foto para fácil entendimento de um usuário leigo e também para um agente da saúde.
\end{itemize}



\section{Justificativa}
O crescimento da população de idosos é um fenômeno mundial e está ocorrendo a um nível sem precedentes. Segundo dados do Instituto Brasileiro de Geografia e Estatística (\sigla{IBGE}{Instituto Brasileiro de Geografia e Estatística}), em 2012, o grupo das pessoas de 60 anos ou mais representava 12,8 porcento da população brasileira residente, porém, em 2017, esse percentual cresceu para 14,6 porcento \cite{jornaldocomercio1}.


Devido às dificuldades de leitura e entendimento das bulas de medicamentos relatadas por leitores e usuários, a Agência Nacional de Vigilância Sanitária (\sigla{ANVISA}{Agência Nacional de Vigilância Sanitária}), órgão brasileiro do Ministério da Saúde, implementou em 19 de janeiro de 2010 a Resolução RDC nº. 47/2009 que estabelece regras para elaboração, harmonização, atualização, publicação e disponibilização de bulas de medicamentos para pacientes e para profissionais da saúde \cite{PORTALANVISA2019}.








