\chapter{Introdução}\label{ch:intro}

As inovações tecnológicas não são apenas artifícios de conforto, também exercem um papel cada vez mais essencial à rotina dos seres humanos. Em particular os idosos, que representam uma crescente parcela significativa da sociedade brasileira, podem se beneficiar do uso das inovações tecnológicas \cite{Goncalves2011}. 

 No Brasil, a queda da taxa de fecundidade, mortalidade e o aumento da longevidade da população vêm contribuindo para o crescente número de idosos com 60 anos ou mais \cite{IBGE2013}. Se, por um lado há o aumento da expectativa de vida e do número de idosos, por outro existe a necessidade de assistência à saúde desta população.
 
 Segundo \citeonline{MURRAY}, o maior consumo de medicamentos é da população idosa, superando todas as outras faixas etárias da população brasileira. Além do tratamento proposto pelo médico, os idosos utilizam produtos não prescritos, ou produtos por engano, tornando então o regime posológico complexo e possibilitam a existência de interações medicamentosas, reações adversas e aumento do custo para o paciente e para a sociedade, levando até a desistência do tratamento correto.
 

Quanto maior o uso contínuo de remédios distintos, maior a possibilidade de surgir questionamentos em relação ao modo correto de uso, função e contra indicação do medicamento. Para um uso correto dos medicamentos, se faz de suma importância as instruções verbais e escritas, claras e objetivas  emitidas pelo médico que muitas das vezes não é assimilada de maneira adequada pelo paciente idoso, fazendo com que a presença de algum material didático para auxiliá-lo em seu tratamento faz-se de fundamental importância \cite{Didonet2007}. No Brasil, a bula representa o principal material informativo fornecido aos pacientes na aquisição de medicamentos produzidos pela indústria farmacêutica.

Segundo \citeonline{FUJITA2006}, a apresentação gráfica do conteúdo informal nas bulas de remédios induzem sua leitura e compreensão. Deficiências, tanto ao nível de conteúdo com uma linguagem muito formal, quanto na apresentação gráfica das informações em bulas, como por exemplo o tamanho da fonte do texto utilizada, pode ser levado ao mau uso de medicamentos, comprometendo a saúde e até acarretando sérias consequências na saúde do indivíduo, principalmente para os idosos, muitas vezes com limitações motoras ou limitados de informação, devido a idade. 


Considerando que cada vez mais os dispositivos móveis terão um papel importante para o acompanhamento de saúde \cite{ESTADAOCELULAR}, a importância da leitura das bulas e da difícil compreensão de seu conteúdo para o uso correto e seguro de medicamentos, o presente trabalho visa identificar padrões em imagens com técnicas de processamento de imagem (\sigla{PDI}{Processamento de Imagem}) e reconhecimento ótico de caracteres (\sigla{OCR}{Reconhecimento Ótico de Caracteres}) embarcado no celular, com intenção de melhorar o poder acertivo de OCR por meio de técnicas de PDI como correção de perspectiva da imagem, determinação da área de interesse, com a finalidade de identificar o medicamento por meio de uma foto capturada da embalagem ou da bula do medicamento, utilizando um telefone celular.

% EVERTON: Seria interessante falar, antes deste último parágrafo, sobre o estado da arte do seu problema. Ou algum trabalho relacionado com PDI e OCR.



\section{Objetivo Geral}

Este trabalho tem como objetivo geral uma melhor acessibilidade à informação medicamentosa em tratamento da população de terceira idade, com o auxilio de técnicas de processamento de imagens e OCR na extração de textos de imagens extraídas por meio de aparelho celular para um aplicativo de auto ajuda.



\section{Objetivos Específicos}

\begin{itemize}
    Os objetivos específicos deste trabalho são:
    \item Especificar o conjunto de técnicas de processamento de imagens que sejam adequadas a imagens de caixas de remédios e bulas de medicamentos.
	\item Propor uma arquitetura para transformar imagens de caixas de remédios e bulas em textos por meio de técnicas de processamento de imagens e OCR.
	\item Analisar qual a eficiência entre o OCR do SDK da Google operando online e offline, aplicados ao problema proposto.
	\item Analisar os resultados obtidos das possíveis técnicas utilizadas na arquitetura proposta.

\end{itemize}


\section{Justificativa}


O acesso da população idosa ao aparelho celular vem se consolidando cada vez mais. O dispositivo celular pode servir como um grande ferramenta em suas tarefas diárias, porém em muitas vezes, a falta de informação, compreensão, limitações motoras e de visão os fazem temer a tecnologia presente em suas mãos. 

No ponto de vista social, este trabalho se justifica em apresentar uma ferramenta simples e fácil para o auxílio na tarefa de polifarmácia de idosos, visando melhor eficácia no tratamento medicamentoso do paciente.

No ponto de vista científico, este trabalho se justifica em apresentar um conjunto de técnicas baseados em PDI, OCR e desenvolvimento de aplicativo de celular que sejam capazes de melhorar o poder acertivo de um reconhecedor ótico de caracteres, viabilizando o reconhecimento de um medicamento por meio de uma foto, tanto da caixa, embalagem ou bula do mesmo.







